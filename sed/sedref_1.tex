\documentclass[a4paper,10pt,landscape,twocolumn]{article}
\usepackage[utf8]{inputenc}
\usepackage[english,russian]{babel}
\usepackage{color}
\usepackage{listings}
\lstloadlanguages{bash}
\lstset{ %
language=bash,                % choose the language of the code
basicstyle=\small\ttfamily\color[gray]{0.40},       % the size of the fonts that are used for the code
commentstyle=,
frame=,			% adds a frame around the code
tabsize=4,			% sets default tabsize to 2 spaces
captionpos=b			% sets the caption-position to bottom
}

\usepackage{verbatim}
\clubpenalty=10000
\widowpenalty=10000

\usepackage{geometry}
\geometry{left=1cm}
\geometry{right=1cm}
\geometry{top=1cm}
\geometry{bottom=1cm}

\pagestyle{empty}

\newcommand{\commanddescr}[2]{\small ~~ \texttt{#1} & \small #2 \smallskip \\ }

\newenvironment{mycommands}[1]
               { \begin{tabular}{lp{9cm}} \\
                   \large{\textbf{#1}} \medskip \\ \hline \\}
               { \\ \hline \end{tabular} \medskip }

\newcommand{\sedex}[1]{\textcolor[gray]{0.40}{\texttt{#1}}}

\begin{document}

\textbf{\Huge{Sed reference}} \\

\begin{mycommands}{Options}
  
  \commanddescr{-f}{
    If you have a large number of sed commands, you 
    can put them into a file and use 

    \sedex{sed -f sedscript <old >new}
  }

  \commanddescr{-n}{ 
    The ``-n'' option will not print anything unless an
    explicit request to print is found 
  }

  \commanddescr{<filename>}{
    You can specify files on the command line if you wish. If there is
    more than one argument to sed that does not start with an option, it
    must be a filename.

    \sedex{sed 's/\^{}\#{}.*//' f1 f2 f3 | grep -v '\^{}\${}' | wc -l}
  }

  \commanddescr{-i}{
    Edit files in place 
  }

  \commanddescr{-r}{
    Use extended regular expressions in the script
  }

\end{mycommands}

\begin{mycommands}{Substitution}
  \commanddescr{s/regex/repl/}{ 
    Substitute regexp with repl 

    \sedex{sed 's/day/night/' <old >new }

    \sedex{sed 's:/usr/local/bin:/common/bin:' <old >new }}

  \commanddescr{\&}{
    Corresponds to the pattern found

    \sedex{\%{} echo ``123 abc'' | sed 's/[0-9][0-9]*/\& \&/'\newline
      123 123 abc}
  }

  \commanddescr{\textbackslash 1}{
    first remembered pattern

    \sedex{sed 's/\textbackslash([a-z]*\textbackslash) \textbackslash([a-z]*\textbackslash)/\textbackslash2 \textbackslash1/'}\newline
    \sedex{sed 's/\textbackslash([a-z]*\textbackslash) \textbackslash1/\textbackslash1/'}
  }
  
  \commanddescr{/g}{
    Global replacement

    \sedex{sed 's/[\^{} ][\^{} ]*/(\&{})/g' <old >new}
  }

  \commanddescr{/2}{
    You can add a number after the substitution command to indicate you only want to match that particular pattern

    \sedex{sed 's/[a-zA-Z]* //2' <old >new}

    \sedex{sed 's/[a-zA-Z]* /DELETED /2g' <old >new}

    \sedex{sed 's/./\&{}:/80' <file >new}

  }
\end{mycommands}

\begin{mycommands}{Substitution}

  \commanddescr{/p}{
    When the ``-n'' option is used, the ``p'' flag will 
    cause the modified line to be printed

    \sedex{sed -n 's/pattern/\&{}/p' <file}
  }
  
  \commanddescr{/w}{
    With it, you can specify a file that will receive the modified data.

    \sedex{sed -n 's/\^{}[0-9]*[02468] /\&{}/w even' <file }
  }

\end{mycommands}

\begin{mycommands}{Restrictions}
  
  \commanddescr{<number>}{
    Restrict command to line number <number>

    \sedex{sed '3 s/[0-9][0-9]*//' <file >new}
  }

  \commanddescr{/<regexp>/}{
    Search for a regular expression 

    \sedex{sed '/\^{}g/s/g/s/g'}

    You can use another delimiter by placing it after \textbackslash{}

    \sedex{sed '\textbackslash{}\_{}/usr/local/bin\_{} s\_{}/usr/local\_{}/common/all\_{}'}
  }

  \commanddescr{<n|re>,<n|re>}{
    You can specify a range on line numbers or regular expressions 
    by inserting a comma
    
    ``\${}'' means last line in the file

    \sedex{sed -e '1,/start/ s/\#{}.*//'}
  }

  \commanddescr{!}{
    Reverse the restriction

    \sedex{sed -n '/match/ !p' </tmp/b}
  }

\end{mycommands}

\begin{mycommands}{Commands}
  
  \commanddescr{d}{
    Delete

    \sedex{sed -e 's/\#{}.*//' -e 's/[ \^{}I]*\${}//' -e '/\^{}\${}/ d' }
  }
  \commanddescr{p}{

    Print

    \sedex{sed -n '1,10 p' <file}

  }
  \commanddescr{q}{
    Quit

    \sedex{sed '11 q'}
  }

\end{mycommands}

 \pagebreak

\begin{mycommands}{Commands}
  \commanddescr{\{{}\}{}}{
    The curly braces, ``\{{}'' and ``\}{}'' are used to group the commands. 
    These braces can be nested, which allow you to combine address ranges.
  }
  &
  \begin{lstlisting}
    #!/bin/sh
    # This is a Borne shell script that 
    # removes #-type comments
    # between 'begin' and 'end' words.
    sed -n '
    1,100 {
      /begin/,/end/ {
        s/#.*//
        s/[ ^I]*$//
        /^$/ d
        p
      }
    }
    '
  \end{lstlisting}
  \\
  \commanddescr{w}{
    Write to a file. Only one space must follow the command.

    \sedex{sed -n '/\^{}[0-9]*[02468]/ w even' <file}
  }

  \commanddescr{r}{
    Read a file

    \sedex{sed '/INCLUDE/ r file' <in >out}

  }

  \commanddescr{a, i, c}{
    The ``a'' command appends a line after the range or pattern.

    \sedex{sed '/WORD/ a\textbackslash{} \newline 
      Add this line after every line with WORD'}

    You can insert a new line before the pattern with the ``i'' command.

    \sedex{sed '/WORD/ i\textbackslash{} \newline
      Add this line before every line with WORD'}

    ``c'' can change the current line with a new line.

    \sedex{sed '/WORD/ c\
      Replace the current line with the line'}
    }

  \commanddescr{=}{
    The ``='' command prints the current line number to standard output.

    \sedex{lines=\`{}sed -n '\${}=' file\`{}}
  }

  \commanddescr{y}{
    Operates like the ``tr'' program.

    \sedex{sed '/0x[0-9a-zA-Z]*/ y/abcdef/ABCDEF' file}
  }
  
  \commanddescr{l}{
    The ``l'' command prints the current pattern space. 
    It also converts unprintable characters into printing characters.
  }
  
\end{mycommands}
  
\begin{mycommands}{Multiple lines}

  \commanddescr{N}{ 
    It reads in the next line and appends a new line character 
    along with the input line itself to the pattern space. 
  }

  \commanddescr{D}{ 
    Deletes the first portion of the pattern space, up
    to the new line character, leaving the rest of the pattern alone.
    If the ``D'' command is executed with a group of other commands in a
    curly brace, commands after the ``D'' command are ignored. The next
    group of sed commands is executed, unless the pattern space is
    emptied.
  }

  \commanddescr{P}{
    Prints the first part of the pattern space, up to the NEWLINE character. 
  }
  &
  \begin{lstlisting}
    #!/bin/sh
    sed '
    /ONE/ {
      # append the next line
      N
      # look for "ONE" followed by "TWO"
      /ONE.*TWO/ {
        # delete everything between
        s/ONE.*TWO/ONE TWO/
        # print
        P
        # then delete the first line
        D
      }
    }' file
  \end{lstlisting}
  \\
\end{mycommands}

\begin{mycommands}{Hold buffer}
  
  \commanddescr{x}{
    eXchanges the pattern space with the hold buffer.
  }

  \commanddescr{h, H}{
    Copy/append pattern space to hold space.
  }

  \commanddescr{g, G}{
    Copy/append hold space to pattern space.
  }


\end{mycommands}
\pagebreak
\begin{mycommands}{Flow control}
  
  \commanddescr{b <label>}{
    Branch to <label>; if label is omitted, branch to end of script.
  }  
  &
  \begin{lstlisting}
    #!/bin/sh
    sed -n '
    # if an empty line, check the paragraph
    /^$/ b para
    # else add it to the hold buffer
    H
    # at end of file, check paragraph
    $ b para
    # now branch to end of script
    b
    # this is where a paragraph 
    # is checked for the pattern
    :para
    # return the entire paragraph
    # into the pattern space
    x
    # look for the pattern, if there - print
    /'\$1'/ p
    '
  \end{lstlisting}
  \\

  \commanddescr{t <label>}{
    The command ``t <label>'' will branch to the <label> if the last 
    substitute command modified the pattern space. 
}
  &
  \begin{lstlisting}
    #!/bin/sh
    sed '
    :again
    s/([ ^I]*)//g
    t again
    '
  \end{lstlisting}
  \\

  
\end{mycommands}

\pagebreak

\begin{tabular}{*{6}{p{1.4cm}}}
  \multicolumn{6}{l}{\large{\textbf{Command summary}}} \medskip \\ \hline \\

 Command
 & 
 Address 
 or Range
 & 
 \multicolumn{4}{c}{ Modifications to}
 \\
 & & 
 Input 
 Stream
 & 
 Output 
 Stream
 & 
 Pattern
 Space
 & 
 Hold
 Buffer
 \\
 \hline
 =&         -&         -&       Y&       -&        - \\
 a&	    1&	       -&	Y&	 -&	   - \\
 b&	    2&	       -&	-&	 -&	   - \\
 c&	    2&	       -&	Y&	 -&	   - \\
 d&	    2&	       Y&	-&	 Y&	   - \\
 D&	    2&	       Y&	-&	 Y&	   - \\
 g&	    2&	       -&	-&	 Y&	   - \\
 G&	    2&	       -&	-&	 Y&	   - \\
 h&	    2&	       -&	-&	 -&	   Y \\
 H&	    2&	       -&	-&	 -&	   Y \\
 i&	    1&	       -&	Y&	 -&	   - \\
 l&	    1&	       -&	Y&	 -&	   - \\
 n&	    2&	       Y&	*&	 -&	   - \\
 N&	    2&	       Y&	-&	 Y&	   - \\
 p&	    2&	       -&	Y&	 -&	   - \\
 P&	    2&	       -&	Y&	 -&	   - \\
 q&	    1&	       -&	-&	 -&	   - \\
 r&	    1&	       -&	Y&	 -&	   - \\
 s&	    2&	       -&	-&	 Y&	   - \\
 t&	    2&	       -&	-&	 -&	   - \\
 w&	    2&	       -&	Y&	 -&	   - \\
 x&	    2&	       -&	-&	 Y&	   Y \\
 y&	    2&	       -&	-&	 Y&	   - \\
\hline

\end{tabular}

\end{document}
